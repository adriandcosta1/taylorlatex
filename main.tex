\documentclass[14pt]{extreport}

\usepackage[english]{babel}
\usepackage[utf8x]{inputenc}
\usepackage{amsmath}
\usepackage{amssymb}
\usepackage{graphicx}
\usepackage[makeroom]{cancel}
\usepackage{tabularx}
\usepackage{xcolor}
\usepackage{relsize}

\title{Proof of Function Representation with Taylor Series}
\author{Adrian D'Costa}

\begin{document}
\maketitle

We know that a power series is:
\newline

\begin{align*} f(x) =& a_0 + a_1(x-a) + a_2(x-a)^2 + a_3(x-a)^3 + a_4(x-a)^4 \\&+ a_5(x-a)^5 + ... + a_n(x-a)^{n}... \text{[where }\left|x-a\right| < R\text{]}\end{align*}
\newline

\section{Section (1)}
Now:
$f(a) = a_0$
\newline

$\therefore a_0 = \frac{f^{(0)}(a)}{0!} \text{ where } f^{(0)}(a) = f(a) \text{ and } 0! = 1$
\newline

\section{Section (2)}

Taking the first derivative of f(x):
\newline

$f^{(1)}(x) = a_1 + 2a_2(x-a) + 3a_3(x-a)^2 + 4a_4(x-a)^3 + 5a_5(x-a)^4 + ...$
\newline

$f^{(1)}(a) = a_1.....\text{(i)}$

$a_{1} = \frac{f^{(1)}(a)}{1!}.....\text{[rearranging (i)]}$
\newline
\newline

\section{Section (3)}

Same way taking the second derivative of $f(x)$:
\newline

$f^{(2)}(x) =  2a_2 + 6a_3(x-a) + 12a_4(x-a)^2 + 20a_5(x-a)^3 + ...$
\newline

$f^{(2)}(a) = 2a_2.....\text{(ii)}$
\newline


$ a_2 = \frac{f^{(2)}(a)}{2!}....\text{[rearranging (ii)]}$
\newline
\newline


\section{Section (4)}

Taking third derivative:
\newline

$f^{(3)}(x) =  6a_3 + 24a_4(x-a) + 60a_5(x-a)^2 + ...$
\newline

$f^{(3)}(a) = 6a_3.....\text{(iii)}$
\newline

$a_3 = \frac{f^{(3)}(a)}{3!}....\text{[rearranging (iii)]}$
\newline
\newline



\section{Section (5)}

Taking fourth derivative:
\newline

$f^{(4)}(x) =  24a_4 + 120a_5(x-a) + ...$
\newline

$f^{(4)}(a) = 24a_4.....\text{(iv)}$
\newline

$a_4 = \frac{f^{(4)}(a)}{4!}....\text{[rearranging (iv)]}$
\newline
\newline


\section{Section (6)}
Taking fifth derivative:
\newline

$f^{(5)}(x) =  120a_5 + ...$
\newline

$f^{(5)}(a) = 120a_5.....\text{(iv)}$
\newline

$a_5 = \frac{f^{(5)}(a)}{5!}....\text{[rearranaging (v)]}$
\newline
\newline


By plugging in values of $a_{0}, a_{1}, a_{2}, a_{3}, a_{4} \text{ and } a_5$ into $f(x)$ we get:

\begin{align*}f(x) =&\frac{f^{(0)}(a)(x-a)^0}{0!} + \frac{f^{(1)}(a)(x-a)^1}{1!} + \frac{f^{(2)}(a)(x-a)^2}{2!} + \frac{f^{(3)}(a)(x-a)^3}{3!} \\&+ \frac{f^{(4)}(a)(x-a)^4}{4!} + \frac{f^{(5)}(a)(x-a)^5}{5!} + ...\end{align*} 
\newline
\newline

So the pattern is:

$f(x) = \mathlarger{\mathlarger{\sum}}\limits_{n = 0}^n\frac{f^{(n)}(a)(x-a)^n}{n!}$
\newline


And that's it. That's the Taylor series. 
$\text{[Q.E.D]}$









\end{document}
